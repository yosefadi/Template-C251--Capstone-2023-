\documentclass{DTETI_CP_C251}

% Penambahan Package yang akan Digunakan
\usepackage{graphicx}
\usepackage{setspace}
\usepackage{subfigure}
\usepackage{array}
\usepackage{indentfirst}
\usepackage{titlesec}
\usepackage{lipsum}
\usepackage[utf8]{inputenc}
\usepackage{multirow}
\usepackage{multicol}
\usepackage{caption}
\usepackage{longtable} 
\usepackage{algorithm}
\usepackage{algpseudocode}
% \usepackage[table,xcdraw]{xcolor}

\bibliographystyle{plain} 
\usepackage[numbers,sort&compress]{natbib}

% ------------------------------------------------------------ %
% Awal Dokumen
\begin{document}

% ------------------------------------------------------------ %
% Data Capstone
% Judul Capstone
\judul{Pembuatan \textit{Template} Laporan \textit{Capstone Project} dalam Format LaTeX}
\title{Template Design for Capstone Project Report in LaTeX Format}

% Jenis Dokumen
%% Jenis Dokumen : PERANCANGAN PRODUK DAN SPESIFIKASI

% Kode Dokumen
%% Kode Dokumen : C-251

% Nomor Dokumen (ID Kelompok Capstone)
\NoDok{C\_06}

% Nomor Revisi
\NoRev{02}

% Tanggal Penerbitan Dokumen
%% Otomatis terisi tanggal ketika file LaTeX ini di-compile

% Data Mahasiswa Capstone
%% Format : \MHS{<Nama Lengkap>}{<NIM>}{<Prodi>}{<Alamat Email>}
% Ketua Kelompok
\MHSA{Gilbert Strang}{20/443000/TK/54100}
	  {Teknik Elektro}{gstrang@mail.ugm.ac.id}
% Anggota 1
\MHSB{Ogata Katsuhiko}{20/443001/TK/54101}
	  {Teknik Biomedis}{kogata@mail.ugm.ac.id} 
% Anggota 2
\MHSC{Randall D. Knight}{20/443010/TK/54110}
	  {Teknologi Informasi}{rdknight@mail.ugm.ac.id}
% Anggota 3
\MHSD{John G. Proakis}{20/443011/TK/54111}
	  {Teknologi Informasi}{jgproakis@mail.ugm.ac.id}
% Anggota 4
\MHSE{Dimitris G. Manolakis}{20/443100/TK/54200}
	  {Teknik Elektro}{dgmanolakis@mail.ugm.ac.id}
% Un-comment Line di bawah ini apabila tidak ada Anggota 4 :
% \MHSE{}{}{}{}

% Dosen Pembimbing
%% Format : {<Nama Lengkap>}{<NIP/NIU>}
\DPA{Carl Friedrich Gauss, B.Eng., M.Eng., D.Eng.}{111 1993 01 2024 01 101}

% Tempat Pelaksanaan
%% Format : {<Nama Laboratorium> \newline <Nama Departemen> \newline <Nama Fakultas>}
\Tempat{Departemen Teknik Elektro dan Teknologi Informasi \newline Fakultas Teknik}
% ------------------------------------------------------------ %
% Membuat Halaman Judul hingga Catatan Revisi Dokumen
\maketitle

% ------------------------------------------------------------ %
% Isi Laporan

% BAB 01 : Pengantar 
\chapter{\uppercase{Pengantar}}
\label{chap:Pengantar}
\input{../Isi_Laporan/BAB_01.tex}

% BAB 02 : Dasar Teori Pendukung 
\chapter{\uppercase{Dasar Teori Pendukung}}
\label{chap:Dasar_Teori_Pendukung}
\input{../Isi_Laporan/BAB_02.tex}

% BAB 03 : Analisis Studi Pustaka Kunci 
\chapter{\uppercase{Analisis Studi Pustaka Kunci}}
\label{chap:Analisis_Studi_Pustaka_Kunci}
\input{../Isi_Laporan/BAB_03.tex}

% BAB 04 : Pemodelan Permasalahan 
\chapter{\uppercase{Pemodelan Permasalahan}}
\label{chap:Pemodelan_Permasalahan}
\input{../Isi_Laporan/BAB_04.tex}

% BAB 05 : Pemilihan dan Pengembangan Metode 
\chapter{\uppercase{Pemilihan dan Pengembangan Metode}}
\label{chap:Pemilihan_dan_Pengembangan_Metode}
\input{../Isi_Laporan/BAB_05.tex}


% BAB 06 : Luaran dan Spesifikasi yang Diusulkan 
\chapter{\uppercase{Luaran dan Spesifikasi yang Diusulkan}}
\label{chap:Luaran_dan_Spesifikasi_yang_Diusulkan}
\input{../Isi_Laporan/BAB_06.tex}

% BAB 07 : Batasan Permasalahan 
\chapter{\uppercase{Batasan Permasalahan}}
\label{chap:Batasan_Permasalahan}
\input{../Isi_Laporan/BAB_07.tex}

% BAB 08 : Perancangan Umum Sistem 
\chapter{\uppercase{Perancangan Umum Sistem}}
\label{chap:Perancangan_Umum_Sistem}
\input{../Isi_Laporan/BAB_08.tex}

% BAB 09 : Rencana Anggaran dan Jadwal Kegiatan
\chapter{\uppercase{Rencana Anggaran dan Jadwal Kegiatan}}
\label{chap:RAB_JadwalKegiatan}
\input{../Isi_Laporan/BAB_09.tex}

% BAB 10 : Simulasi Pendahuluan 
\chapter{\uppercase{Simulasi Pendahuluan}}
\label{chap:Simulasi_Pendahuluan}
\input{../Isi_Laporan/BAB_10.tex}

% BAB 11 : Kesimpulan 
\chapter{\uppercase{Kesimpulan}}
\label{chap:Kesimpulan}
\input{../Isi_Laporan/BAB_11.tex}

% ------------------------------------------------------------ %
% References

\begin{thebibliography}{}

    \input{Referensi.tex}

\end{thebibliography}

% Akhir Dokumen
\end{document}