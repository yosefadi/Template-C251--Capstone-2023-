% Judul Capstone
\judul{Pembuatan \textit{Template} Laporan \textit{Capstone Project} dalam Format LaTeX}
\title{Template Design for Capstone Project Report in LaTeX Format}

% Jenis Dokumen
%% Jenis Dokumen : PERANCANGAN PRODUK DAN SPESIFIKASI

% Kode Dokumen
%% Kode Dokumen : C-251

% Nomor Dokumen (ID Kelompok Capstone)
\NoDok{C\_06}

% Nomor Revisi
\NoRev{02}

% Tanggal Penerbitan Dokumen
%% Otomatis terisi tanggal ketika file LaTeX ini di-compile

% Data Mahasiswa Capstone
%% Format : \MHS{<Nama Lengkap>}{<NIM>}{<Prodi>}{<Alamat Email>}
% Ketua Kelompok
\MHSA{Gilbert Strang}{20/443000/TK/54100}
	  {Teknik Elektro}{gstrang@mail.ugm.ac.id}
% Anggota 1
\MHSB{Ogata Katsuhiko}{20/443001/TK/54101}
	  {Teknik Biomedis}{kogata@mail.ugm.ac.id} 
% Anggota 2
\MHSC{Randall D. Knight}{20/443010/TK/54110}
	  {Teknologi Informasi}{rdknight@mail.ugm.ac.id}
% Anggota 3
\MHSD{John G. Proakis}{20/443011/TK/54111}
	  {Teknologi Informasi}{jgproakis@mail.ugm.ac.id}
% Anggota 4
\MHSE{Dimitris G. Manolakis}{20/443100/TK/54200}
	  {Teknik Elektro}{dgmanolakis@mail.ugm.ac.id}
% Un-comment Line di bawah ini apabila tidak ada Anggota 4 :
% \MHSE{}{}{}{}

% Dosen Pembimbing
%% Format : {<Nama Lengkap>}{<NIP/NIU>}
\DPA{Carl Friedrich Gauss, B.Eng., M.Eng., D.Eng.}{111 1993 01 2024 01 101}

% Tempat Pelaksanaan
%% Format : {<Nama Laboratorium> \newline <Nama Departemen> \newline <Nama Fakultas>}
\Tempat{Departemen Teknik Elektro dan Teknologi Informasi \newline Fakultas Teknik}