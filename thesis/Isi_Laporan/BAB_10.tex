Pada bagian ini, apabila diperlukan Anda dapat menyertakan simulasi pengantar. Simulasi pengantar ini tidak harus berupa program simulasi. Sebagai contoh dalam perancangan perangkat keras, peserta dapat menyampaikan simulasi perangkat keras yang di desain dengan menggunakan perangkat lunak bawaan (\textit{tools}) misalkan Orchad/PSpice untuk mensimulasikan bagian-bagian dari keseluruhan desain misalnya jika desain berupa perangkat keras lengkap dimana bagian-bagian perangkat keras tersebut terdapat \textit{low-pass filter}, \textit{high-pass filter}, ADC (\textit{Analog-to-Digital Converter}), dan DAC (\textit{Digital-to-Analog Converter}), maka Anda perlu mensimulasikan komponen-komponen \textit{low-pass filter},\textit{ high-pass filter,} ADC, dan DAC sesuai dengan desain yang diinginkan. Misalnya peserta akan mendesain \textit{low-pass filter} Butterworth dengan frekuensi \textit{cut-off} 100 Hz, maka peserta perlu merancang komponen $R$ dan $C$ serta mensimulasikan dengan perangkat lunak yang tersedia.

Jika mensimulasikan operasi sistem tenaga listrik, peserta bisa menggunakan perangkat lunak yang biasa dipakai misalkan Digsilent, ETAP dan lain-lain. Jika model yang dihasilkan berupa \textit{model multi machines} yang terinterkoneksi dengan beban, maka Anda dapat mendesain simulasi berdasarkan perangkat lunak (misalnya MATLAB) menjadi bagian-bagian kecil dari sistem tersebut. Anda bisa mensimulasikan satu generator yang dipakai jika disambungkan ke \textit{infinite bus}, selain itu bisa mensimulasikan saluran transmisi, mensimulasikan beban-beban aktif dan reaktif, mensimulasikan motor, mensimulasikan PV, dan sebagainya.
