Pada bagian ini peserta \textit{capstone} menguraikan lebih lanjut spesifikasi dari rancangan produk di atas. Spesifikasi yang dimaksud di sini adalah spesifikasi yang bersifat fungsional. Tim juga harus mendefinisikan batasan-batasan permasalahan secara lebih terperinci dan konkret. Batasan-batasan ini diperlukan untuk membuat sistem yang dirancang bisa diimplementasikan secara masuk akal dalam jangka waktu, kesediaan sarana dan prasarana, jumlah sumber daya manusia (SDM), biaya yang terbatas, dan lain-lain. 

Pada bidang TI, peserta \textit{capstone} dapat menjelaskan fasilitas atau fitur apa saja yang disediakan oleh sistem yang dirancang secara lebih detail. Sebagai contoh, sistem informasi keuangan dapat dispesifikasikan untuk memiliki kemampuan untuk melakukan perhitungan biaya atau gaji (\textit{payroll}) berserta laporan (\textit{reporting}). Pada kasus ini, peserta wajib menyebutkan secara detail jenis pelaporan yang ada. Kemudian, perlu juga dijelaskan bagaimana perhitungan tersebut dilakukan, serta \textit{user} apa saja yang dikehendaki terlibat dalam sistem yang dirancang. Di samping itu, peserta wajib mencantumkan apa saja keterbatasan sistem yang dirancang, misalnya fitur apa saja yang tidak dapat atau tidak akan dilakukan atau tidak disediakan pada produk yang akan dikembangkan. 

Pada bidang TE yang menghasilkan perangkat keras, dapat dijelaskan mengenai fitur fungsional dan karakteristik dari perangkat keras tersebut. Sebagai contoh, misalkan sebuah rancangan robot diharapkan dapat digunakan untuk membantu orang tua yang memiliki keterbatasan. Dalam memenuhi tugas tersebut, robot dapat dispesifikasikan memiliki kriteria tertentu, misalkan, stabil pada kondisi medan yang homogen maupun heterogen (lapangan rumput, aspal, con-block). Selain, itu robot dapat dispesifikasikan memiliki bentuk yang kokoh dan rapi, pengkabelan (\textit{wiring}) tidak berantakan, tertutup dan aman. Perlu disebutkan juga keterbatasan robot tersebut, misalnya hanya membahas robot beroda dan tidak bisa bergerak \textit{omni-directional}, daya yang tersedia maksimal adalah 40 Watt, dan sebagainya. 

Pada bidang TE yang menghasilkan program simulasi, maka peserta \textit{capstone} dapat menspesifikasikan jenis program yang akan dibuat. Misalkan, sebuah program simulasi yang dibuat untuk menjalankan algoritme dalam rangka menjawab sebuah permasalahan (tentu saja permasalahan yang akan dipecahkan telah dijelaskan terlebih dahulu). Di samping juga batasan-batasan dari algoritme yang akan dirancang. Di samping itu, peserta \textit{capstone} secara umum harus menspesifikasikan skenario yang dipakai dalam proses perancangan perangkat lunak diatas. Pengujian bisa mespesifikasikan metrik/indikator yang akan diuji. Pada bidang STL yang tidak menghasilkan perangkat keras maupun program simulasi lengkap, maka peserta \textit{capstone} wajib menspesifikasikan analisis kebutuhan dan detil permasalahan yang akan dijawab, beserta batasan-batasan yang diprediksi mungkin timbul beserta alasan-alasan yang logis kenapa batasan-batasan tersebut dipilih/ditetapkan.
